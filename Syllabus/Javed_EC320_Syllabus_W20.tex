% This syllabus template was created by:
% Brian R. Hall
% Associate Professor, Champlain College
% www.brianrhall.net

% Document settings
\documentclass[11pt]{article}
\usepackage[margin=1in]{geometry}
\usepackage[pdftex]{graphicx}
\usepackage{multirow}
\usepackage{setspace}
\pagestyle{plain}
 \usepackage{float} 
\newcommand{\ra}[1]{\renewcommand{\arraystretch}{#1}}
\setlength\parindent{0pt}
\usepackage{booktabs}
\usepackage{hyperref}
\usepackage[dvipsnames]{xcolor}

\begin{document}

\begin{center}
\textbf{\Large EC 320: Introduction to Econometrics\\
M/W Fenton 110, 4-5:20 pm }
\end{center}

\vspace{0.2in}

% Course information
 \textbf{\large Amna Javed }
  \hfill  
 \textbf{ \large GE: Sichao Jiang} \\  
 \large Office: PLC 520
 \hfill 
 \large Office: PLC 525\\
  \large Email: amnaj@uoregon.edu 
 \hfill 
 \large sichaoj@uoregon.edu \\ 
 \large OH: W 2-3:30 pm 
 \hfill 
  \large OH: T/R 10-11 am

\bigskip

 \textbf{\large GE: Cory Briar}
  \hfill  
 \textbf{ \large GE: Youssef Ait Benasser} \\  
 \large Office: PLC 523
 \hfill 
 \large Office: PLC 506\\
  \large Email:  cbriar@uoregon.edu
 \hfill 
 \large   youssefa@uoregon.edu \\ 
 \large OH: M/W 1-2 pm
 \hfill 
 
\vspace{5mm}

\begin{center}  This syllabus is preliminary and is subject to change.  \\
\end{center}

% Course details

\section*{Course summary}

\textbf {\large \\ Description:} This course introduces the statistical techniques that help economists learn about the world using data. We will focus much of our attention on regression analysis, the workhorse of applied econometrics. Using calculus and introductory statistics, we will cultivate a working understanding of the theory underpinning regression analysis---\textit{how} it works, \textit{why} it works, and, \textit{when it can lead us astray}. We will apply the insights of theory to work with and learn from actual data using \texttt{{R}}, a statistical programming language. To the extent that you invest the requisite time and effort, you can leave this course with marketable skills in data analysis and---most importantly---a more sophisticated understanding of the notion that \textbf{correlation does not necessarily imply causation}. 


\textbf {\large \\ Prerequisites:} Math 242 (Calculus) and Math 243 (Introduction to Statistics) or equivalent.

\subsection*{Software}

\begin{itemize}
  \item We will use the statistical programming language \href{https://www.r-project.org/}{\textbf{\texttt{R}}}.
  \item We will use \href{https://www.rstudio.com}{\textbf{\texttt{RStudio}}} to interact with \texttt{R}.
\end{itemize}
Learning \texttt{R} is challenging, but well worth the effort. \texttt{R} is a powerful and versatile tool for data analysis and visualization, which makes it popular among employers. The SSIL lab in McKenzie has \texttt{R} and \texttt{RStudio} installed and ready for you, but I strongly recommend that you install these programs on your own computer. Don't worry, \textbf{both are free}. I also recommend that you purchase a flash drive to save your scripts, data, and assignments. Alternatively, you can use the class network drive (the ``R drive"), which is available on all university computers.
\newpage 
If you are concerned about learning \texttt{R}---or you want to learn quicker---I recommend that you check out the following free online resources:
\begin{itemize}

  \item \href{https://r4ds.had.co.nz/index.html}{ \textit{\textcolor{PineGreen}{R for Data Science}}}
  \item \href{http://adv-r.had.co.nz/}{Hadley Wikham's \textit{\textcolor{PineGreen}{Advanced R}}}
    \item \href{https://www.econometrics-with-r.org/index.html}{\textit{\textcolor{PineGreen}{ Introduction to Econometrics with R}}}
\end{itemize}

\subsection*{Textbooks}

\paragraph{Required:} There is one required textbook for this course:

\begin{enumerate}
	\item \href{http://smile.amazon.com/Introduction-Econometrics-Christopher-Dougherty/dp/0199676828/}{\textbf{Introduction to Econometrics}, 5\textsuperscript{th} ed.} by Christopher Dougherty
\end{enumerate}
You can purchase this at the Duckstore or your preferred online bookseller. You should complete the assigned readings from the textbooks \textit{before} lecture. Attending lecture is not a substitute for reading and comprehending the texts. Likewise, reading is not a substitute for attending lecture. The lectures and the readings are meant to \textit{complement} one another. The tentative course schedule (further below) lists the assigned readings for each topic.

In addition to the textbook readings, I may occasionally assign readings from peer-reviewed studies for classroom discussion. I will post these readings on Canvas.

\section*{Course Structure}

\subsection*{Grading}

I will award grades based on your relative performance in the class, as determined by the following weights:
\begin{table}[!h]
	
	\centering
	\begin{tabular}{@{\extracolsep{1cm}}ll@{}}
		\textbf{Problem Sets} & 30\% \\
		\textbf{Midterm Exam I} & 20\% \\
		\textbf{Midterm Exam II} & 20\% \\
		\textbf{Final Exam}   & 30\%
	\end{tabular}
\end{table}

\subsection*{Problem Sets} 

I will assign \textbf{five} problem sets throughout the quarter. Each problem set will include an analytical component and a computational component. 
\begin{itemize}
	\setlength{\itemsep}{0pt}
	\item I will announce due dates in class. 
	\item You will turn in an \textbf{electronic copy} of each problem set on Canvas.
	\item Presentation matters. 
	\item I will drop your lowest problem set score.
\end{itemize}
I encourage you to work together on the problem sets. Unless explicitly stated, \textbf{each student is required to write and submit independent answers}. I will take word-for-word copies as evidence of academic dishonesty. If you work with others, list their names at the top of your assignment. If you fail to list your collaborators, you will receive a score of zero.

\subsection*{Exams} 

During the exams, you may use a writing utensil and a non-programmable calculator. As you turn in your exam, I will ask you to present your student ID. I do not give makeup exams. See the course policy on makeup assignments for more information.  

\subsection*{Lab} 

In your weekly lab section, you will learn to apply the concepts discussed in lecture using \texttt{R}. While the lab may include some general econometrics instruction, the main focus is on the practical application of statistical techniques and working through the computational components of the problem sets. Attending lab is crucial for learning the material and passing the course. Due to space constraints, \textbf{you must attend the lab for which you registered}.

\section*{Course Policies}

\subsection*{Late Policy} 

If you turn in a problem set after the deadline, you will receive a 2 percentage point deduction for each \textbf{hour} your assignment is late. If you turn it in after I post the key, you will receive a zero.

\subsection*{Makeup Assignments} 

I do not give makeup assignments or exams. Please check the final exam schedule before registering for this course. In extreme circumstances that lead you to miss one of the midterm exams---such as death in the family or grave illness or injury---I will consider re-weighting your grade toward the final. To qualify for re-weighting, you will need to notify me no later than two days after the exam.

\subsection*{Grade Appeals} 

You must submit any request for re-grading in writing \textbf{within one week} of the day grades are posted for the problem set or exam in question. Your request should include a cogent argument explaining why your responses warrant more credit.

\subsection*{Etiquette} 

Please respect those around you by turning off your phone and other potentially distracting devices. I ask that you stay for the entire lecture: getting up and leaving distracts your fellow classmates. If you must leave early, please position yourself near the door when you get to class. 

\subsection*{Academic Integrity} 

I will not tolerate cheating, plagiarism, and other violations of the \href{https://studentlife.uoregon.edu/conduct}{Student Conduct Code}. If I catch you cheating or plagiarizing on any component of this course, you will receive a failing grade for the term and I will report your offense to the university. 

\subsection*{Accommodations} 

Notify me if there are aspects of this course that pose disability-related barriers to your participation. If you require special accommodations for a documented disability, then you will need to provide me a letter from the \href{https://aec.uoregon.edu/}{Accessible Education Center} (AEC) that verifies your need and details the appropriate accommodations. Please make arrangements with the AEC by the end of Week 1. If your accommodations include exam proctoring at the AEC, then you are responsible for scheduling those exams with the AEC \textit{at least seven days in advance}.

\subsection*{Discrimination and Harassment Policy} 

I will direct students who are not minors and who report sexual violence or sexual harassment to me to resources to help them and will report to the university administration only when requested by the student (unless someone is in imminent risk of serious harm). Students experiencing any form of prohibited discrimination or harassment, including sex, gender, race, or religion based violence, may seek information on safe.uoregon.edu, respect.uoregon.edu, or investigations.uoregon.edu or contact the non-confidential Title IX office (541-346-8136), Office of Civil Rights Compliance (541-346-3123), or Dean of Students offices (541-346-3216), or call the 24-7 hotline 541-346-SAFE for help. I am also a mandatory reporter of child abuse. Please find more information at Mandatory Reporting of Child Abuse and Neglect.
\newpage
\section*{Tentative Schedule}

\begin{table}[H]
	\caption*{\large\textbf{Lectures and Exams}}
	\centering
  \ra{1.5}
  \begin{tabular}{@{\extracolsep{0.5cm}} c c l l @{}}
    \toprule
    \textbf{Week} & \textbf{Date} & \textbf{Topic} & \textbf{Reading}  \\ \toprule
    01 & 06/01 & Introduction & \\
    02 & 08/01 & Statistics Review I & ItE Review \\
    02 & 13/01 & Statistics Review II & ItE Review \\
    03 & 15/01 & The Fundamental Econometric Problem & \\
    03 & 20/01 & \textit{No class} &  \\
    03 & 22/01 & The Logic of Regression &  \\
    04 & 27/01 & Carry forward and Midterm I Review & \\ \midrule
    04 & 29/01 & \textbf{Midterm Exam I} (in-class) \\ \midrule
    05 & 03/02 & Simple Linear Regression: Estimation I & ItE 1  \\
    05 & 05/02 & Simple Linear Regression: Estimation II & ItE 1 \\
    06 & 10/02 & Classical Assumptions & ItE 1  \\
    06 & 12/02 & Simple Linear Regression: Inference & ItE 2 \\
    07 & 17/02 & Multiple Linear Regression: Estimation & ItE 3 \\
    07 & 19/02 & Multiple Linear Regression: Inference & ItE 3 \\  \midrule
    08 & 24/02 & \textbf{Midterm Exam II} (in-class) \\ \midrule
     08 & 26/02& Categorical Variables & ItE 5  \\
    09 & 02/03& Interactive Relationships & ItE 4  \\
    09 & 04/03& Nonlinear Relationships & ItE 4  \\
    10 & 09/03 & TBD &   \\
    10 & 11/03 & Final Review &\\
    11 & 16/03 & \textbf{Final Exam} (see \href{https://registrar.uoregon.edu/calendars/examinations#complete-final-exam-schedule}{\textcolor{PineGreen}{final exam schedule}}) \\
    \bottomrule
  \end{tabular}
\end{table}

\begin{center}
	\textbf{Subject to change!}
\end{center}


\end{document}



