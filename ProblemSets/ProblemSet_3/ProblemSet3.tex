


\documentclass[11pt]{article}
\usepackage[margin=1in]{geometry}
\usepackage{fancyhdr}
\usepackage{amsmath , amsthm , amssymb}
\usepackage{graphicx}
\usepackage{hyperref}
\usepackage{authblk}
\usepackage{setspace}

\usepackage{lscape} %this is to make the landscape of individual table pages.
\usepackage{enumitem, float, booktabs}
\setcounter{MaxMatrixCols}{10}

\pagestyle{fancy}
\lhead{}
\chead{}
\rhead{}
\lfoot{}
\cfoot{\thepage}
\rfoot{}
\setlength{\parindent}{0.5in}
\geometry{left=.8in,right=1in,top=1in,bottom=1in}
\renewcommand{\baselinestretch}{2}
\begin{document}

\begin{onehalfspacing}

\begin{center}
\textbf{EC 320: Introduction to Econometrics} \bigskip

\textbf{Problem Set\bigskip\ 3}
\bigskip
\end{center}


\noindent \textbf{Total: 50 points}

\noindent \textbf{Due: Tuesday January 11th, 2020 at 1 pm}

\bigskip

\noindent \textbf{Learning Outcomes:}
\begin{itemize}
\item Understanding regression analysis  
\item Understanding how to do basic proofs
\item Understanding omitted variable bias
\end{itemize}

\bigskip


\noindent \textbf{Checklist Before Handing In:}
\begin{itemize}
\item Did you answer all questions?
\item Did you answer all parts for each question?
\item Were your answers too vague? If so, make them more precise to make sure they really answer the question being asked.
\end{itemize}

\bigskip

\noindent {\textbf{Instructions:}}\ You are encouraged to work with other students in the class, but you must provide original responses. To receive full credit, justify your answers and list your collaborators. For full credit on the computational exercises, include your code and output in addition to your answers. You will turn in digital copies of your responses on Canvas. Please note the list of acceptable file types on the submission page.  \\
\vspace{0.1in}

Name: 			\\
\vspace{0.1in}

Collaborator 1: \\

\vspace{0.1in}

Collaborator 2: \\
	
\vspace{0.1in}

Collaborator 3: 		
			

\newpage


\begin{center}
\textbf{Analytical Questions} \bigskip
\end{center}

\begin{enumerate}

\item Consider the data on hours studied per week and final grades in EC 320 from a sample of four students: (2 points each)

\begin{table}[htb]
	\centering
	\begin{tabular}{@{\extracolsep{1cm}} c c c @{}}
		\toprule
		\textbf{Student} & \textbf{Hours Studied} & \textbf{Final Grade} \\ \toprule
		Hinata & 7 & 72\% \\
		Michelle & 10 & 97\% \\
		Jerry & 4 & 43\% \\
		Jorge & 6 & 84\% \\
		\bottomrule
	\end{tabular}
\end{table}


\begin{enumerate}
	\item  Suppose that you want to use the data to learn about the effect of studying on grades. Write down a simple linear regression model tailored to your objective. 

	\item Calculate the parameter estimates for the intercept ($\hat{\beta}_1$) and slope ($\hat{\beta}_2$) using the OLS formulas from class.

	\item Interpret $\hat{\beta}_1$. What does this parameter estimate tell us?

	\item Interpret $\hat{\beta}_2$. What does this parameter estimate tell us?

	\item Calculate the coefficient of determination ($R^2$) for the regression you estimated. What does it tell us about the relationship between studying and grades?

	\item What is the predicted final grade for a student who studies 5 hours per week?

	\item Based on the regression you estimated, how many hours would a student have to study to expect a score of 100\%?

\end{enumerate}

\item Consider the predicted values of $Y_i$ from a simple linear regression of $Y_i$ on $X_i$, which we refer to as $\hat{Y}_i$. Prove that the sample mean of $\hat{Y}_i$ is equal to the sample mean of $Y_i$ (i.e. $\bar{Y}$). (3 points)

\item Consider the residuals ($\hat{u}_i$) from a simple linear regression. Prove that the sample mean of $\hat{u}_i$ is equal to zero. (3 points)

\item Assume that $\bar{Y}$ and $\bar{X}$ are equal to zero. Prove that the sample covariance of $X_i$ and $\hat{u}_i$ is equal to zero. (3 points)

\item Suppose that you run a regression of $Y_i$ on $X_i$ and obtain parameter estimates $\hat{\beta}_1$ and $\hat{\beta}_2$. Then, using the same data, you decide to run a regression of $\tilde{Y}_i$ on $X_i$, where $\tilde{Y}_i = 2Y_i$. Prove that both regressions have the same $R^2$. (3 points)

\end{enumerate}

\newpage 

\begin{center}
\textbf{Computational Questions} \bigskip
\end{center}

\noindent For this portion of the problem set, you will use the \texttt{fertility.csv} excel file in the Problem Set 3 folder on Canvas. The file contains a random sample of data on the fertility and education of women interviewed in the Demographic and Health Survey in India. To complete this assignment, you will need to load the \texttt{tidyverse} and \texttt{stargazer} package. \\

\noindent The Demographic and Health Survey interviews women and asks them information about their fertility and health behavior. The variables in the data set are described below.\\

\begin{table}[htb]
	\centering
	\begin{tabular}{@{\extracolsep{1cm}} l l @{}}
		\toprule
		\textbf{Variable Name} & \textbf{Description}  \\ \toprule
		\texttt{id} & Unique id of each woman interviewed \\
		\texttt{educ} & Years of schooling completed by the woman (respondent) \\
		\texttt{peduc} & Years of schooling completed by the partner \\
		\texttt{kids} & Number of children \\
		\texttt{agem} & Age of marriage for the woman \\
		\bottomrule
	\end{tabular}
\end{table}

\noindent Economic theory suggests that education can decrease fertility. One hypothesized mechanism is that education improves job prospects, which increases the opportunity cost of having children. Another hypothesized mechanism---particularly relevant in developing countries---is that education increases awareness of effective contraceptive methods. We will delve into the effect of parental education on fertility behavior in India. (2 points each)


\begin{enumerate}
	
\item Does marrying later in life reduce fertility, as measured by the number of children? 
\begin{enumerate}
	\item Write down a simple linear regression model that describes the effect of age at marriage on fertility. Use variable names that relate to the research question.
	\item Using the \texttt{lm()} function, estimate the simple linear regression model that you specified above.
	\item  Make a regression table using \texttt{stargazer}. Make sure you specify \texttt{type = "html"}. To reduce clutter, pass \texttt{keep.stat = c("rsq", "n")} as an argument to \texttt{stargazer()}.
	\item Visualize your regression results by making a scatter plot with a fitted regression line. \textbf{Hint:} Use \texttt{stat\_smooth(method = "lm", se = FALSE)} in your \texttt{ggplot} code.
	\item  Interpret the intercept estimate. Is this reasonable?
	\item  Interpret the slope estimate.
	\item Is the slope coefficient statistically significant? 
	\item Does the slope estimate warrant causal interpretation? In other words, do you think that the slope describes the causal effect of education on fertility? Why or why not? If not, identify a potential source of selection bias or omitted-variable bias.
\end{enumerate}

\item Does education reduce fertility, as measured by the number of children? 
\begin{enumerate}
	\item Write down two simple linear regression models that describe the effect of parent's education on fertility. Use variable names that relate to the research question.
	\item Run two separate simple linear regressions that allow you to estimate the effect of mother's education and father's education on fertility. Use one independent variable per regression. Summarize the results in a regression table.
	\item Identify and interpret the $R^2$ from the regression of fertility on father's education.
	\item Of the three independent variables you considered, which one is the best predictor of fertility? Justify your answer. 
\end{enumerate}
\end{enumerate}


\end{onehalfspacing}
\end{document}
