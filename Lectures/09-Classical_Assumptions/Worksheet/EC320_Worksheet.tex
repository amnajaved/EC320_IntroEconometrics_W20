% This syllabus template was created by:
% Brian R. Hall
% Associate Professor, Champlain College
% www.brianrhall.net

% Document settings
\documentclass[11pt]{article}
\usepackage[margin=1in]{geometry}
\usepackage[pdftex]{graphicx}
\usepackage{multirow}
\usepackage{setspace}
\pagestyle{plain}
\usepackage{enumerate}
\usepackage[margin=1in]{geometry}
\usepackage{fancyhdr}
\usepackage{amsthm , amssymb}
\usepackage{graphicx}
\usepackage{hyperref}
\usepackage{authblk}
\usepackage{setspace}

\usepackage{lscape} %this is to make the landscape of individual table pages.
\usepackage{float, booktabs}
\usepackage{amsmath}

\setlength\parindent{0pt}

\begin{document}


\indent \textbf{Exercise: Classical Assumptions}\\


\begin{itemize}
\item Consider the following regression: 
\begin{equation*}
Consumption_i = \frac{\beta_1}{\beta_2} Income_i + \mu_i
\end{equation*}
Is this linear in parameters? Do we meet the OLS assumptions?

\vspace{0.35in}

\item Consider the following regression: 
\begin{equation*}
Consumption_i = \beta_1 + \beta_2 (Income_i)\mu_i
\end{equation*}
Is this linear in parameters? Do we meet the OLS assumptions?

\vspace{0.35in}

\item Consider the following regression: 
\begin{equation*}
Grades_i = \beta_1 + \beta_2 (Hours Spent Studying_i) + \mu_i
\end{equation*}
You run this analysis using retrospective administrative data. Are you likely to meet the assumption that E(u$\mid$X) = 0? Why or why  not?
\vspace{0.35in}


 \textbf{Exercise: Classical Assumptions}\\

\item Consider the following regression: 
\begin{equation*}
Consumption_i = \frac{\beta_1}{\beta_2} Income_i + \mu_i
\end{equation*}
Is this linear in parameters? Do we meet the OLS assumptions?

\vspace{0.35in}

\item Consider the following regression: 
\begin{equation*}
Consumption_i = \beta_1 + \beta_2 (Income_i)\mu_i
\end{equation*}
Is this linear in parameters? Do we meet the OLS assumptions?


\vspace{0.35in}


\item Consider the following regression: 
\begin{equation*}
Grades_i = \beta_1 + \beta_2 (Hours Spent Studying_i) + \mu_i
\end{equation*}
You run this analysis using retrospective administrative data. Are you likely to meet the assumption that E(u$\mid$X) = 0? Why or why  not?
\vspace{0.35in}

\vspace{0.35in}


\item Consider the following regression: 
\begin{equation*}
Grades_i = \beta_1 + \beta_2 (Hours Spent Studying_i) + \mu_i
\end{equation*}
You run this analysis using data collected from a randomized control trial, where you make some students spend more time studying by delaying their test by 2 hours. Are you likely to meet the assumption that E(u$\mid$X) = 0? Why or why  not?
\vspace{0.35in}

\item  Consider the following regression: 
\begin{equation*}
GDP_t = \beta_1 + \beta_2 (Total Labor Activity_t) + \mu_t
\end{equation*}
where t is time (month or the year), and Total Labor Activity represents the total hours put in by the country's workforce. Which OLS assumption(s) are likely violated in this example?

\vspace{2in}


\item Consider the following regression: 
\begin{equation*}
Grades_i = \beta_1 + \beta_2 (Hours Spent Studying_i) + \mu_i
\end{equation*}
You run this analysis using data collected from a randomized control trial, where you make some students spend more time studying by delaying their test by 2 hours. Are you likely to meet the assumption that E(u$\mid$X) = 0? Why or why  not?
\vspace{0.35in}

\item  Consider the following regression: 
\begin{equation*}
GDP_t = \beta_1 + \beta_2 (Total Labor Activity_t) + \mu_t
\end{equation*}
where t is time (month or the year), and Total Labor Activity represents the total hours put in by the country's workforce. Which OLS assumption(s) are likely violated in this example?

\end{itemize}


\end{document}


